\documentclass[11pt,toc=sectionentrywithoutdots, headheight=44pt, headings=optiontoheadandtoc, hyperfootnotes=false]{scrartcl}

%Packages
\usepackage{geometry}
\usepackage[utf8]{inputenc}
\usepackage{csquotes}
\usepackage{times}
\usepackage{verbatim}
\usepackage[T1]{fontenc}
\usepackage[ngerman]{babel}
\usepackage{graphicx}
\usepackage[headsepline]{scrlayer-scrpage}
\usepackage{blindtext}
\usepackage{footnotebackref}
\usepackage{nameref}
\usepackage{setspace}

\usepackage[
backend=biber,
style=authortitle,
citestyle=authoryear
]{biblatex}

\addbibresource{literatur.bib}

\usepackage{tikz,ifthen,xstring,calc,pgfkeys,pgfopts}
\usepackage{tikz-uml}
\usetikzlibrary{external}
\tikzexternalize[prefix=figures/] % activate and define figures/ as cache folder

\usepackage{hyperref}
\usepackage[printonlyused, withpage]{acronym}

%%%%%%%%%%%%%%%%%% Compile optimization - Kill for release!!!%%%%%%%%%%%%%%
% Kill draft in Dokumentclass!!!
%\pdfcompresslevel=0
%\pdfobjcompresslevel=0
%%%%%%%%%%%%%%%%%%%%%%%%%%%%%%%%%%%%%%%%%%%%%%%%%%%%%%%%%%%%%%%%%%%%%%%%

\hypersetup{
    colorlinks=true,
    linktoc=page,
    linkcolor=blue,
    filecolor=magenta,      
    urlcolor=cyan,
    citecolor=blue
}



%%%%%%%%%%%%%% Set line Spacing for Abbreviation %%%%%%%%%%%%%%%%%%
%\renewenvironment{description}
%{\list{}{\labelwidth0pt\itemindent-\leftmargin
%    \parsep0pt\itemsep0pt\let\makelabel\descriptionlabel}}
%               {\endlist}
%%%%%%%%%%%%%%%%%%%%%%%%%%%%%%%%%%%%%%%%%%%%%%%%%%%%%%%%%%%%%%%%%%%%%%

%\usetikzlibrary{external}
%\tikzexternalize %

%Function for getting sectionName---------------------------------------------------

\newcounter{secautolabel}
\AddtoDoHook{heading/endgroup}{\setautolabel}
\newcommand*{\setautolabel}[1]{%
  \stepcounter{secautolabel}%
  \label{sec:autolabel:\thesecautolabel}%
  \expandafter\xdef\csname #1title\endcsname{%
    \noexpand\nameref*{sec:autolabel:\thesecautolabel}%
  }%
}

\newcommand\sectionRefs{%
	\sectiontitle  
}


%Function for getting sectionName----------------------------------------------------



%Images path
\graphicspath{{./src/images/}}

%Document meta data
\geometry{bottom=25mm, right=25mm, left=25mm, top=35mm}

%Set page numbering font
\renewcommand{\headfont}{\normalfont}

 
\begin{document}



\thispagestyle{empty}
\cfoot[]{}
%\ofoot{Seite \thepage\:von \pageref{LastPage}}
\ofoot{\thepage}
\ifoot{© Frank Loleit}
\ohead{\includegraphics[scale=0.125]{argusLogo}}



\ihead
{%	
	\begin{small}%
		MASCHINELLE ANALYSE VON SOCIAL MEDIA POSTS%		
		\newline\
		\textit\sectionRefs
	\end{small}%
}


%IHK-Logo
\begin{figure}[h]
\includegraphics[scale=0.25]{ihkLogo}
\centering
\end{figure}

\begin{center}
\begin{Large}

Abschlussprüfung Winter 2021
\linebreak

Fachinformatiker (Anwendungsentwicklung)\linebreak
Dokumentation der betrieblichen Projektarbeit
\linebreak\linebreak
\end{Large}


\begin{Huge}
\begin{bfseries}
Maschinelle Analyse von\linebreak Social Media Posts
\linebreak\linebreak
\end{bfseries}
\end{Huge}

\begin{Large}
Backend-Applikation zur algorithmusbasierten 
Erfassung von\linebreak Meldungen in sozialen Netzwerken
\linebreak

Abgabetermin: 22.11.2021
\linebreak\linebreak
\begin{bfseries}
Auszubildender:\linebreak
\end{bfseries}
Frank Loleit\linebreak
Wildenbruchstr. 43\linebreak
12435 Berlin\linebreak

\begin{figure}[h]
\includegraphics[scale=0.25]{argusLogo}
\centering
\end{figure}



\begin{bfseries}
Ausbildungsbetrieb:\linebreak
\end{bfseries}
ARGUS DATA INSIGHTS® Deutschland GmbH\linebreak
Gneisenaustr. 66\linebreak
10961 Berlin


\end{Large}


\end{center}





\newpage
\setcounter{page}{1}
\pagenumbering{Roman}


\tableofcontents


\setcounter{secnumdepth}{0}

\newpage

\addcontentsline{toc}{section}{Abbildungsverzeichnis}
\listoffigures




\newpage

\section{Tabellenverzeichnis}
\blindtext\blindtext\blindtext

\section{Listings}
\blindtext\blindtext\blindtext
\newpage

\section{Abkürzungsverzeichnis}


\begin{acronym}[xxxxxxx]

\acro{ADI}{Argus Data Insights Deutschland GmbH}
\acro{GUI}{Graphical User Interface}
\acro{XAML}{Extensible Application Markup Language}

\end{acronym}

\newpage

\section{Begriffserläuterung}
\blindtext\blindtext
\newpage

%\setcounter{secnumdepth}{1}
\pagenumbering{arabic}
\setcounter{secnumdepth}{4}
\ofoot{Seite\:\thepage\:von \pageref*{myLastPage}}

\section{Einleitung}
\blindtext\blindtext

\subsection{Projektbeschreibung}
Softwareentwicklung ist eine tolle Sache. So noch ein Satz\footnote{vgl. \cite{Gettner2015}}.
Der Computer arbeitet dann wie von alleine und man muss kaum noch etwas machen\footnote{vgl. \cite{Theis2014}}.

\subsection{Projektziel}
Ein wichtiges Ziel ist es, die \acs{GUI} so zu gestalten, dass am Ende keine Unstimmigkeiten mit dem 
Webinterface auftreten. Die \acs{ADI} ist daher bemüht weitere Schritte einzuleiten. Man hat sich entschieden \acs{XAML} zu verwenden. Auch Mathematisch ist dies wichtig\footnote{vgl. \cite{GeraldTeschl2008}}.

\subsection{Projektumfeld}
Hier ist natürlich die Schach-KI entscheidend\footnote{vgl. \cite{Schrepel2011}}.

\subsection{Projektbegründung}
Die Gründe für mein Projekt sind einfach. Ich wurde von der heiligen Mutter auserwählt,
um das Volk der Fachinformatiker zu erleuchten. So ist diese Arbeit entstanden.

\subsection{Projektabgrenzung}
Wichtig ist natürlich die logische Planung\footnote{vgl. \cite{Weigel2015}}.%
Aber auch die IT-Systeme spielen eine Rolle\footnote{vgl. \cite{Gettner2015}}.

\section{Projektplanung}
\blindtext\

\subsection{Projektphasen}
\blindtext

\subsection{Ressourcenplanung}
\blindtext\

\subsection{Entwicklungsprozess}
\blindtext\

\section{Analysephase}
\blindtext\

\subsection{IST-Zustand}
\blindtext\

\subsection{Wirtschaftlichkeitsanalyse}
\blindtext\

\subsubsection{Beschaffung der Ausgangsdaten}
\blindtext\

\subsubsection{Ermitteln der Kostenstruktur}
\blindtext\

\subsubsection{Kostenberechnung}
\blindtext\

\subsubsection{Zusammenstellen des Kostenplans}
\blindtext\

\subsubsection{Nicht-monetäre Vorteile}
\blindtext\

\subsection{Anwendungsfälle}
\blindtext\

\subsection{Lastenheft}
\blindtext\

\subsection{Risikomanagement}
\blindtext\

\subsubsection{Risikoerkennung}
\blindtext\

\subsubsection{Risikoanalyse}
\blindtext\

\subsubsection{Risikominimierung}
\blindtext\

\section{Entwurfsphase}
\blindtext\

\subsection{Zielplattform}
\blindtext\

\subsection{Architekturdesign}
\blindtext\

\subsection{Entwurf des Datenmodells}
\blindtext\

\subsection{Entwurf Geschäftslogik}
\blindtext\

\subsubsection{Entwurf Geschäftslogik}
\blindtext\

\subsubsection{Entwurf Modul Extraktion}
\blindtext\

\subsubsection{Entwurf Modul Laden}
\blindtext\

\subsubsection{Entwurf Modul Transformation}
\blindtext\

\subsection{Roundtrip Engineering}
\blindtext\

\subsection{Pflichtenheft}
\blindtext\

\section{Implementierung}
\blindtext\

\subsection{Implementierung Extraktion}
\blindtext\

\subsection{Implementierung Laden}
\blindtext\

\subsection{Implementierung der Datenstrukturen}
\blindtext\

\subsection{Implementierung Transformation}
\blindtext\

\subsection{Implementierung Hilfsklassen und Interfaces}
\blindtext\

\section{Qualitätssicherung}
\blindtext\

\subsection{Systemtest}
\blindtext\

\subsection{Tests gegen das System}
\blindtext\

\subsection{Testprotokolle}
\blindtext\

\section{Abnahme}
\blindtext\

\subsection{Soll/Ist-Vergleich}
\blindtext\

\subsection{Abnahme durch den technischen Leiter}
\blindtext\

\section{Dokumentation}
\blindtext\

\subsection{Technische Dokumentation}
\blindtext\

\subsection{Nutzerdokumentation}
\blindtext\

\section{Fazit}
\blindtext\

\subsection{Im Laufe des Projektes gewonnene Erkenntnisse}
\blindtext\

\subsection{Ausblick}
\blindtext\

\subsubsection{Von Anfang an geplante Weiterentwicklung}
\blindtext\

\subsubsection{Im Laufe des Projektes erkannter Weiterentwicklungsbedarf}
\blindtext\


\newpage
\setcounter{secnumdepth}{0}
\addcontentsline{toc}{section}{Literatur}

\printbibliography



\label{myLastPage}

\newpage

\pagenumbering{roman}

\setcounter{page}{1}
\setcounter{secnumdepth}{2}
\setcounter{section}{1}
\renewcommand{\thesection}{A}


\ofoot{\thepage}




\section{Anhang}
\subsection{Detaillierte Zeitplanung}
\blindtext

\subsection{Ressourcenplan}
\blindtext\blindtext

\subsection{Projektkosten Details}
\blindtext\blindtext

\subsection{Anwendungsfalldiagramm}
\blindtext\blindtext

\subsection{Lastenheft (Auszug)}
\blindtext\blindtext

\subsection{Quell- und Zieldatenschema}
\blindtext\blindtext

\subsection{Sequenzdiagramm}
\blindtext
\clearpage

\subsection{Klassendiagramm}

Dies ist ein Beispiel für ein UML-Klassendiagramm:\\

\begin{figure}[ht]
\centering
\begin{tikzpicture}

	\umlclass[x=5, y= 0]{Person}	
		{
			- vorerkrankungen : String\\
			- religion : string\\
			+ alter : int\\
			+ name : String\\
			+ geschlecht : String\\
			
		}
		{
			+ halloSagen() : void		\\	
			+ freundlichLaecheln() : void		\\	
		}
		
		
	\umlclass[x=0, y= -6]{Kind}	
		{
			- kinderGarten: String\\
			+ hobbys : String \\
			+ magNutella : bool \\
		}
		{
			- inDieSchuleGehen() : void	\\
			- dreiradFahren() : void\\
			+ getKindergarten() : String\\
			
		}	
		
		
	\umlclass[x=10, y= -6]{Erwachsener}	
		{
			- beruf: String\\
			- kontostand : double\\
			+ hatKinder : bool\\
		}
		{
			- zurArbeitGehen() : void	\\
			- autoFahren() : void\\
			+ getBeruf() : String\\
			
		}
		
		\umlnote{Person}{Dies ist offensichtlich eine Klasse}		
		\umlinherit[geometry=|-]{Erwachsener}{Person}
		\umlinherit[geometry=-|]{Kind}{Person}
		\umldep[very thick, arg1=0..*, arg2=1..*, pos1=0.05, pos2=0.52,anchor1=-20, anchor2=-20]{Kind}{Erwachsener}

	
\end{tikzpicture}

\caption{Klassendiagramm}
\label{fig:Klassendiagramm}
	
\end{figure}

\subsection{Pflichtenheft (Auszug)}
\blindtext

\subsection{Laufzeitdiagnose}
\blindtext

\subsection{Ganttdiagramm}
\blindtext

\subsection{Projektstrukturplan}
\blindtext

\subsection{Arbeitspakete}
\blindtext

\subsection{Arbeitspaketressourcen im Detail}
\blindtext

\end{document}